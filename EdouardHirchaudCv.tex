\documentclass[a4paper, 11pt]{article}

%%%%%%%%%%%%%%%%%%%%%%%%%%%%%%%%%%%%%%%%%%%%% 
\usepackage[utf8]{inputenc} 
\usepackage[francais]{babel} 
\usepackage{graphicx}
\usepackage[francais,detail]{simple-cv-style} 
\usepackage[francais]{layout}
%%%%%%%%%%%%%%%%%%%%%%%%%%%%%%%%%%%%%%%%%%%%%%



%%%%%%%%%%%%%%%%%%
\begin{document} 


\Identity{Edouard Hirchaud} 
{1 rue Émile Péhant} 
{44000 Nantes} 
{06-03-05-20-64} 
{edouard.hirchaud@univ-nantes.fr/ed.hirchaud@gmail.com} 
{32 ans : 12 Janvier 1982} 

%%%%%%%%%%%%%%%%%%%%%%%%%%%%%%%%%%%%%%%%%%%%%%%%%%%%%%%%%%%%%%%%%%%%%%%%%%%
%http://www.lesia.obspm.fr/perso/florence-henry/CoursLatex/courslatex.php %
%%%%%%%%%%%%%%%%%%%%%%%%%%%%%%%%%%%%%%%%%%%%%%%%%%%%%%%%%%%%%%%%%%%%%%%%%%%


\TitleFunction{ Ingénieur d'étude en Bioinformatique} 

\Summary{ }
 
%%%%%%%%%%%%%%%%%%%%%%%%%%%%%%%%%%%%%% 
\begin{Section}{Experiences professionelles et stages}
\Work{Plateforme Bioinformatique de Nantes : INSERM }
{Janv. 2010-actuel}
{CDD Ingenieur d'étude Bioinformatique\\ Richard Redon, Audrey Bihouée}
{Analyse de données : Prestation d'analyse en puce à ADN (Agilent/Affymetrix), transcriptomique\\
	Développement informatique : 
	\begin{itemize}
		\item LIMS (interface web/base de données)
		\item Package R pour l'analyse et la génération de rapport pour les puces ADN.
		\item Développment d'outils pour Galaxy
	\end{itemize}
	Élaboration de Formation : \begin{itemize}
		\item Initiation à R (3 sessions)
		\item	Analyse de donnée de puce avec R (3 sessions)
		\item 	Analyse de données de puce avec Galaxy (En développement)
	\end{itemize}
	Norme ISO9001 : Responsable processus gestion de projet en génomique
}

\Work{Institut Pasteur (Paris) : PF4 Intégration et Analyses de génomes}
{Sep. 07-Juil. 09}
{Mission professionnelle en alternance (18 mois) \\ Encadrants Catherine Dauga, Pierre Dehoux}
{Analyse comparative des protéases à Sérine issues de sept génomes d'insectes par approche phylogénétique
 \\
 Développement de PhyloClust, un logiciel pour la comparaison des grandes familles protéiques}

\Work{Institut Pasteur (Paris) : Biochimie et Biologie Moléculaire des Insectes}
{Fév. 07-Juil. 07} 
{Stage de Master (4 mois). \\  Encadrants : Karin Eiglmeier, Catherine Dauga} 
{Comparaison in silico des CDS de protéases à Sérine d'Anopheles gambiae de deux versions de génomes (Moz2a et AgamP3)\\
Vérification de prédictions de gènes en laboratoire}

\end{Section}
 %%%%%%%%%%%%%%%%%%%%%%%%%%%%%%%%%%%%%
\begin{Section}{Formation Universitaire}

\YearStudy{Master 2 Santé Recherche et Professionnel de Bioinformatique (ex-Dess : Egoist) }
{Université de Rouen}
{ }
{2007-2009}

\YearStudy{Master 1 en Biochimie spécialité Bioinformatique }
{Université de Rouen}
{ }
{2007}

 \YearStudy{Licence en Biochimie}
{Université de Rouen}
{ }
{2006}

 \YearStudy{DEUG en Science de la Vie}
{Université de Rouen}
{ }
{2005}

 \YearStudy{Pharmacie 1ere année}
{Université de Rouen}
{ }
{2003}

\end{Section}
%%%%%%%%%%%%%%%%%%%%%%%%%%%%%%%%%%%%%
\begin{Section}{Compétences}
 
\SubPoint BioInformatique :
\begin{List}
\item Analyses de Séquences : recherche de séquence par Blast, alignement de séquences (simple ou multiple)
\item Banques de données publiques (NCBI, ENSEMBL, GOLD, PDB, GEO) 	
\end{List}

\SubPoint Statistiques :
\begin{List}
\item Logiciel R et bibliothèques de Bioconductor
\item Statistiques descriptives
\item Analyse multivariés.
\end{List}	

\SubPoint Informatique :  
\begin{List}
\item Langage de script (Perl/BioPerl, bash/ Ruby / Groovy)
\item Langage orienté objet (Java, Javadoc, JUnit) 
%\item good knowledge of Perl,
\item Bases de données relationelles : MySQL, SQL  
\item Interfaçage Web : HTML/CSS, ,JEE, Spring, Grails)
\item Gestion de version : Git
\item Logiciel gestion de projet : Redmine.
\item Systèmes d'exploitation : Linux, Mac, Windows(7)	
\item Administration de serveur Linux.
\end{List}

\SubPoint Bureautique :
\begin{List}
\item {\LaTeX Markdown}
\item Suite LibreOffice/Office
\end{List}

\end{Section}
%%%%%%%%%%%%%%%%%%%%%%%%%%%%%%%%%%%%% 
\begin{Section}{Langue}
 
	\Language{écrit, lu, parlé}
\end{Section} 
%
%\begin{Section}{Personal Interest}
%
%\SubPoint{Sport} : Swinmming
%
%\SubPoint{Music} : Rock musik, Play Basse guitar, Computer assisted composition (open softwar)
%
%
%\SubPoint{Other} : Photographics film, Read, watch films.
%
%\end{Section}
%%%%%%%%%%%%%%%%%%%%%%%%%%%%%%%%%%%%%
\begin{Section}{Interêt personnel}

\SubPoint{Sport} : Natation, course à pied

\SubPoint{Musique} : Musique rock, pratique basse électrique, contrebasse

\SubPoint{Littérature} : Fantasy, Science Fiction.

\SubPoint{Jeux de société} : Jeux de gestion, stratégie et diplomatie.

\end{Section}

%%%%%%%%%%%%%%%%%%%%%%%%%%%%%%%%%%%%%
%\begin{Section}{Références}
%\Reference{Dr Catherine Dauga}
%{Institut Pasteur Paris}
%{PF4 : Intégration et analyse des génomes}
%{Rue Docteur Roux}
%{75015 Paris}
%{c.dauga@pasteur.fr}
%{01 45 68 87 46}
%\Reference{Maître de Conférence Hélène Dauchel}
%{UFR des Sciences et Techniques}
%{LITIS EA 4108 : Laboratoire d'Informatique Traitement de l'Information et des Systèmes}
%{Université de Rouen}
%{76821 Mont-St-Aignan}
%{Helene.Dauchel@univ-rouen.fr}
%{02 35 14 63 89}
%\end{Section}

%%%%%%%%%%%%%%%%%%%%%%%%%%%%%%%%%%%%%
\end{document} 
