\documentclass[11pt]{letter}

\usepackage[utf8]{inputenc}
\usepackage[T1]{fontenc}
\usepackage[french]{babel}

\makeatletter
\newcommand*{\NoRule}{\renewcommand*{\rule@length}{0}}
\makeatother
\name{Edouard Hirchaud}
\address{478 rue du Cdt Dubois 76230 Bois-Guillaume}
\email{ed.hirchaud@gmail.com}
\date{24 juillet 2009}
\francais


\begin{document}

\begin{letter}{Céline Brochier-Armanet \\ David Moreira}
\NoRule
%	\def\concname{Objet :~}
%	\conc{Candidature : Offre de bourse de thèse}
	\opening{Madame Brochier-Armanet, Monsieur Moreira,}

	Le projet de thèse que vous proposé m'intéresse tout particulièrement, car il
  correspond à la direction que je souhaite donner à mes études et serait de
  nature à me fournir l'opportunité de mettre en pratique et d'approfondir mes
  connaissances acquises durant mon stage de master.

  Je viens de terminer mon master 2 de bioinformatique effectué à l'université
  de Rouen, qui se déroule en deux ans durant lesquels j'ai effectué un stage
  en alternance (15 mois de stage et 7 mois de cours) dans l'unité
  d'Intégration et d'Analyse Génomique de l'Institut Pasteur de Paris. J'y ai
  étudié les protéases à Sérine de différents insectes (drosophile,
  moustiques\ldots) afin d'en améliorer l'annotation.

	Dans ce cadre j'ai établi et automatisé, par développement informatique, une
	démarche de ''clustering'' phylogénétique. J'ai acquis de ce fait de solides
	connaissances dans l'utilisation et la compréhension d'outil d'alignement de
	séquences, multiples ou deux à deux (algorithme de Smith \& Watterman, Blast,
	T-Coffe, Promals, Clustal) et de reconstruction phylogénétique (méthode de
	distances, maximum de vraisemblance, inférence bayésienne, parcimonie). Ces
	connaissances sont complétées par les différents cours suivis pendant le master, dans
	les domaines de génomiques, métagénomique, analyses statistiques.


	 Pendant ces deux ans j'ai non seulement appris des aspects techniques en
	 bioinformatique, mais surtout confirmer mon  gout pour la recherche,
   notamment en évolution.


  Je reste à votre disposition pour tous entretiens que vous
  jugerez nécessaire.


	\closing{Je vous prie, Madame, Monsieur, de bien vouloir accepter
	l'expression de mes salutations distinguées.} 
\end{letter}

\end{document}
